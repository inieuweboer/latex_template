\documentclass[a4paper]{article}
\usepackage{structure/mystyle}



\title{Latex template}
\author{Ismani Nieuweboer}


\begin{document}
  \maketitle


% $$
% \nabla_{\mathbf{w}_j} \boldsymbol{\mathcal{L}^{(n)}}
% = \sum_{i} \frac{ \partial \boldsymbol{\mathcal{L}^{(n)}} }{\partial W_{ij}} e_i
% = \sum_{i} \frac{ \partial \boldsymbol{\mathcal{L}^{(n)}} }{\partial \log(q_j)} \frac{ \partial \log(q_j) }{\partial W_{ij}} e_i
% = \sum_{i} \delta_j^q \frac{ \partial \log(q_j) }{\partial W_{ij}} e_i
% = \delta_j^q  \nabla_{\mathbf{w}_j} \log(q_j)
% = \delta_j^q  \mathbf{h}
% $$


% \begin{restatable}[Euclid]{theorem}{firsteuclid}
% \label{thm:euclid}
% For every prime \(p\), there is a prime \(p’>p\).
% In particular, the list of primes,
% \begin{equation}\label{eq:1}
% 2,3,45,7,\dots
% \end{equation}
% is infinite.
% \end{restatable}





% \firsteuclid*
% \vdots
% \firsteuclid*



% \Autoref{thm:euclid}


\end{document}



